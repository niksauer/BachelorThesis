%!TEX root = ../main.tex

\addchap{\langabkverz}
%nur verwendete Akronyme werden letztlich im Abkürzungsverzeichnis des Dokuments angezeigt
%Verwendung: 
%		\ac{Abk.}   --> fügt die Abkürzung ein, beim ersten Aufruf wird zusätzlich automatisch die ausgeschriebene Version davor eingefügt bzw. in einer Fußnote (hierfür muss in header.tex \usepackage[printonlyused,footnote]{acronym} stehen) dargestellt
%		\acs{Abk.}   -->  fügt die Abkürzung ein
%		\acf{Abk.}   --> fügt die Abkürzung UND die Erklärung ein
%		\acl{Abk.}   --> fügt nur die Erklärung ein
%		\acp{Abk.}  --> gibt Plural aus (angefügtes 's'); das zusätzliche 'p' funktioniert auch bei obigen Befehlen
%	siehe auch: http://golatex.de/wiki/%5Cacronym
%	

% acronym settings
\makeatletter
\newif\if@in@acrolist
\AtBeginEnvironment{acronym}{\@in@acrolisttrue}
\newrobustcmd{\LU}[2]{\if@in@acrolist#1\else#2\fi}

\newcommand{\ACF}[1]{{\@in@acrolisttrue\acf{#1}}}
\makeatother

% start acronyms
\begin{acronym}[YTMMM]
\setlength{\itemsep}{-\parsep}

\acro{ABI}{Application Binary Interface}
\acro{API}{Application Programming Interface}
\acro{dApp}{Decentralized Application}
\acro{DPG}{Deutsche Pfandsystem GmbH}
\acro{EAN}{European Article Number}
%\acro{EOA}{Externally Owned Account}
\acro{EOA}{\LU{E}{E}xternally \LU{O}{o}wned \LU{A}{a}ccount}
\acro{ERC}{Ethereum Request for Comments}
\acro{EVM}{Ethereum Virtual Machine}
\acro{HPE}{Hewlett Packard Enterprise}
\acro{LIFO}{Last In, First Out}
\acro{NPM}{Node Package Manager}
\acro{QoS}{Quality of Service}
\acro{URI}{Uniform Resource Identifier}

\end{acronym}

%!TEX root = ../main.tex

\pagestyle{empty}

\iflang{de}{%
% Dieser deutsche Teil wird nur angezeigt, wenn die Sprache auf Deutsch eingestellt ist.
\renewcommand{\abstractname}{\langabstract} % Text für Überschrift

% \begin{otherlanguage}{english} % auskommentieren, wenn Abstract auf Deutsch sein soll
\begin{abstract}
%Abstract normalerweise auf Englisch. Siehe:  \url{http://www.dhbw.de/fileadmin/user/public/Dokumente/Portal/Richtlinien_Praxismodule_Studien_und_Bachelorarbeiten_JG2011ff.pdf} (8.3.1 Inhaltsverzeichnis)
%
%Ein "`Abstract"' ist eine prägnante Inhaltsangabe, ein Abriss ohne Interpretation und Wertung einer wissenschaftlichen Arbeit. In DIN 1426 wird das (oder auch der) Abstract als Kurzreferat zur Inhaltsangabe beschrieben.
%
%\begin{description}
%\item[Objektivität] soll sich jeder persönlichen Wertung enthalten
%\item[Kürze] soll so kurz wie möglich sein
%\item[Genauigkeit] soll genau die Inhalte und die Meinung der Originalarbeit wiedergeben
%\end{description}
%
%Üblicherweise müssen wissenschaftliche Artikel einen Abstract enthalten, typischerweise von 100-150 Wörtern, ohne Bilder und Literaturzitate und in einem Absatz.
%
%Quelle: \url{http://de.wikipedia.org/wiki/Abstract} Abgerufen 07.07.2011
%
%Diese etwa einseitige Zusammenfassung soll es dem Leser ermöglichen, Inhalt der Arbeit und Vorgehensweise
%des Autors rasch zu überblicken. Gegenstand des Abstract sind insbesondere 
%\begin{itemize}
%\item Problemstellung der Arbeit,
%\item im Rahmen der Arbeit geprüfte Hypothesen bzw. beantwortete Fragen,
%\item der Analyse zugrunde liegende Methode,
%\item wesentliche, im Rahmen der Arbeit gewonnene Erkenntnisse,
%\item Einschränkungen des Gültigkeitsbereichs (der Erkenntnisse) sowie nicht beantwortete Fragen. 
%\end{itemize}
%Quelle: \url{http://www.ib.dhbw-mannheim.de/fileadmin/ms/bwl-ib/Downloads_alt/Leitfaden_31.05.pdf}, S.~49
%\end{abstract}
% \end{otherlanguage} % auskommentieren, wenn Abstract auf Deutsch sein soll
}



\iflang{en}{%
% Dieser englische Teil wird nur angezeigt, wenn die Sprache auf Englisch eingestellt ist.
\renewcommand{\abstractname}{\langabstract} % Text für Überschrift

\begin{abstract}
%What is the problem?
%What was done?
%What was discovered?
%What do the findings mean?

Germany's deposit-refund system has missed its opportunity to halt the influx of one-way packaged beverages. Shockingly, the system even benefits bottlers as they are allowed to pocket all non-claimed deposits. New measures will be needed to increase the portion of reusable bottles and thus support the environment through the smaller ecological footprint offered.

This thesis introduces the concept of an incentivized deposit-refund system, in which consumers are rewarded for choosing reusable beverage packaging but also penalized for throwing away one-way bottles. Several of its core features speak for a blockchain-based approach. To this end, the research question is as follows: \textit{What would such an implementation look like, and are there any special considerations to be made?} 

This question is answered by carrying out the end-to-end development process which includes design, implementation and testing. Moreover, the study aims to highlight the major obstacles companies must expect when developing applications for decentralized platforms. Although the results indicate that its capabilities are sufficient, the platform does not currently scale enough to support this particular workload.


\end{abstract}
}
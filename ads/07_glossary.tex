%!TEX root = ../main.tex

%
% vorher in Konsole folgendes aufrufen:
%	makeglossaries makeglossaries dokumentation.acn && makeglossaries dokumentation.glo
%

%
% Glossareintraege --> referenz, name, beschreibung
% Aufruf mit \gls{...}
%

\newglossaryentry{beverage packaging}{
	name={beverage packaging},
	description={Predominantly closed packaging for foods of liquid nature intended for consumption as a drink, excluding yogurt and kefir \cite[§~3]{verpackV1991}}
}

\newglossaryentry{reusable packaging}{
	name={reusable packaging},
	description={Packaging intended to be reused for the same purpose after having been used. Characterised by having set up the logistics to take back, clean and refill the packaging. The sole intention or claim to be reused is not valid \cite[§~3]{verpackV1991}}
}

\newglossaryentry{ecologically advantageous packaging}{
	name={ecologically advantageous packaging},
	description={Packaging that does now show any significant ecological disadvantages when compared to reusable packaging \cite[pp.~83]{Flanderka/Stroetmann2009}}
}

%
%\newglossaryentry{scripting language}{
%	name={scripting language},
%	description={A domain-specific programming language that will only execute in special run-time environments. Examples include: Bash for UNIX, JavaScript for web browsers or Visual Basic for Applications \cite{scriptinglanguage}} 
%}
%
%\newglossaryentry{software unit}{
%	name={software unit},
%	description={Piece of software created during the lifecycle of a software system. It is relevant for development, operation and maintenance of that software system. Can be developed, maintained and replaced independently from other units. A unit is atomic in the sense of handling multiple units},
%	plural={software units},
%}
%
%\newglossaryentry{configuration}{
%	name={configuration},
%	description={Set of software units associated to each other that build-up a (part-)system}
%}
%
%\newglossaryentry{configuration management}{
%	name={configuration management},
%	description={Role or organisational unit that exclusively identifies, administrates and offers software units. It controls and documents changes to the unit} 
%}
%
%\newglossaryentry{version}{
%	name={version},
%	description={An initial release or re-release of a document, as opposed to a revision resulting from issuing change pages to a previous release \cite{24765-2017}},
%	plural={versions},
%}
%
%\newglossaryentry{variant}{
%	name={variant},
%	description={Variants are created out of a version and exists in parallel. Examples include customising software for specific needs, such as adopting it to different operating systems},
%	plural={variants},
%}
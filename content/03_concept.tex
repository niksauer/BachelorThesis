%!TEX root = ../main.tex

\chapter{Concept}
\label{chp:concept}

\section{Solution Overview}

%- Usability (Rechenzeit, Storage, Dauer, Kosten)
%- Design Ansätze > Backlog, Wartezeit (Vergleich mit Internetspeed damals), Transaktionsminimierung
%- Public oder Private Blockchain?
%- Abrechnung pro Monat, da ansonsten kein Prozentsatz ermittelt werden kann
%- funktional > User Stories
%- nicht-funktional > schnelle Antwortzeiten, günstige Transaktionen, Skalierbarkeit 


An incentivized deposit-refund system, as is outlined in the following, aims to:

\begin{enumerate}[label=(\Alph*)]
  \item \label{bolsterQuote} bolster the quote of beverages sold in reusable packaging
  \item \label{preventPollution} prevent pollution of the environment caused from throwing away bottles
  \item support the cause of environmental agencies
\end{enumerate}

From a game theoretical perspective, \ref{bolsterQuote} and \ref{preventPollution} are presumably achieved by \footnote{Lending itself to the fact that solely threatening retailers and bottlers with the introduction of a deposit-refund system had not been a successful approach (comp. \ref{itm:levyDepositObligation} and \autoref{fig:reusableQuoteDevelopment}).}:

\begin{itemize}
  \item rewarding consumers who purchase reusable bottles
  \item punishing consumers whose bottles are eventually thrown away
\end{itemize}

To translate these measures into reality, it will be necessary to alter the current deposit-refund system (see \ref{sec:depositRefundCycle}) so that:

\begin{itemize}
  \item non-claimed deposits are redistributed among agencies and reusable bottle consumers
  \item a penalty is added onto future deposits of consumers who repeatedly pollute
\end{itemize}

This revised deposit-refund system is portrayed in \autoref{fig:depositCycleFuture} on \autopageref{fig:depositCycleFuture}.

	
\subsection{Functional Requirements}
	
\subsection{Non-Functional Requirements}
	
\subsection{Derived Assumptions}

%\section{Evaluation Framework}

\section{Architecture}


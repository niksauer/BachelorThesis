%!TEX root = ../main.tex

\chapter{Concept}
\label{chp:concept}

\section{Incentivized Deposit-Refund System}

\subsection{Overview}
\label{sec:incentivizedOverview}
An incentivized deposit-refund system, as is outlined in the following, aims to:

\begin{enumerate}[label=(\Alph*)]
  \item \label{bolsterQuote} bolster the quote of beverages sold in reusable packaging
  \item \label{preventPollution} prevent pollution of the environment caused from throwing away bottles
  \item support the cause of environmental agencies
\end{enumerate}

From a game theoretical perspective, \ref{bolsterQuote} and \ref{preventPollution} are presumably achieved by \footnote{Lending itself to the fact that solely threatening retailers and bottlers with the introduction of a deposit-refund system had not been a successful approach (comp.~\autoref{ftn:presumedEffects} and \autoref{fig:reusableQuoteDevelopment} on pp.~\rangeref{presumedHappenedEffects}).}:

\begin{itemize}
  \item rewarding consumers who purchase reusable bottles
  \item punishing consumers whose bottles are eventually thrown away
\end{itemize}

To translate these measures into reality, it will be necessary to alter the current deposit-refund system (see~\ref{sec:depositRefundCycle}) so that:

\begin{itemize}
  \item non-claimed deposits are redistributed among agencies and reusable bottle consumers
  \item a penalty is added onto future deposits of consumers who repeatedly pollute
\end{itemize}

Practically, this will require garbage collection to report how many and which specific one-way bottles have been thrown away. Otherwise, it will not be possible to calculate the amount of non-claimed deposits and identify which consumer was responsible for the bottle at last. The latter infers that bottles must be uniquely identifiable. On the other hand, such a system can only be realized if retailers forward the information that a consumer has bought reusable bottles and has thereby become eligible to receive rewards. In order to establish proper ownership, they must also report which one-way bottles a consumer has purchased. Naturally, it should be prevented that anybody can report these figures, pretend to be an environmental agency or claim someone else's rewards. This core functionality, together with the existing minimum level of service, can be expressed in terms of the different interactions a certain user may have with the system, known as a use case diagram and given by \autoref{fig:useCaseOverview}.

\begin{figure}[hbt]
  \includegraphics[width=\textwidth]{images/concept/use_case_overview}
  \caption{Use case overview}
  \label{fig:useCaseOverview}
\end{figure}

\FloatBarrier

From this it becomes apparent that a revised approach will essentially incorporate an accounting layer to enable the punishment and rewarding of consumers. Basic protocols of interaction (i.e.~buying and returning a bottle) should only be modified to the minimum extent needed.

\subsection{Proposed Rules}
\label{sec:rules}
Having given a high-level introduction to an incentivized deposit-refund system, this section aims to provide an exemplary set of rules that may constitute the underlying business logic. Again, it must be noted that this only serves as a guideline for architecting such a solution. The precise economics, psychological effectiveness and ethics introduced hereby are disregarded (comp.~\ref{sec:goalsScope}).

\subsubsection{Rewards \& Donations}

\begin{itemize}
 	\item 50\% of non-claimed deposits are reserved for donations (\textit{agency fund})
  	\item 50\% of non-claimed deposits are reserved for rewards (\textit{consumer fund})
  	\item rewards and donations can be claimed once in each period
  	\item a period's duration is set at 4 weeks
  	\item a consumer's reward ($r_c$) is based on the amount of reusable bottles ($b$) he has purchased within the previous period, thus can be calculated as:

\begin{equation}
r_c = \frac{b}{b_{total}} \times f_c
\end{equation}

where $b_{total}$ denotes the total number of reusable bottles sold within that period and $f_c$ resembles the consumer fund

	\item non-claimed rewards are withheld for self-financing
  	\item the agency fund ($f_a$) is evenly distributed, so that each's donation ($d$) will be $d = f_a/n$, where $n$ refers to the number of approved agencies
  	\item non-claimed donations remain available for all agencies in the next period
\end{itemize}

\subsubsection{Penalties}

\begin{itemize}
 	\item if a consumer repeatedly pollutes, he must pay a penalty on top of each deposit
	\item the threshold ($t$) to receive a penalty is set at 5 thrown away bottles
  	\item the penalty's value ($v$) is set at \EUR{0.05}
  	\item a consumer's penalty ($p_c$) is proportional to the number of bottles ($b$) he has thrown away during the lifetime of this system, meaning that: 

\begin{equation}
p_c = \frac{b - (b \mod t)}{t} \times v
\end{equation}

  	\item the penalty is refunded, if the consumer returns the bottle on his own
  	\item the penalty is seized, if the bottle is thrown away or returned by someone else
  	\item seized penalties are also used for self-financing
\end{itemize}

% functional requirements formatter
\def\twodigits#1{%
%  \ifnum#1<100 0\fi
  \ifnum#1<10 0\fi
  \number#1}

\newlist{FR}{enumerate}{1}
\setlist[FR]{label=FR-\arabic*:}

% start of functional requirements
\subsection{Derived Functional Requirements}
In accordance to \ref{sec:incentivizedOverview} and \ref{sec:rules}, the following functional requirements are defined as the bare minimum of features any implementation must provide in order to be used as the accounting backend of an incentivized deposit-refund system. These requirements are expressed from the perspective of an end-user who hopes to achieve different goals by using the system and are sorted by the various roles encountered (comp.~\autoref{fig:useCaseOverview}).

\paragraph{Garbage collector}
\begin{enumerate}[label={\textbf{FR-\protect\twodigits{\theenumi}}},leftmargin=*]
	\item As a garbage collector, I can report the number of thrown away bottles, so that the amount of non-claimed deposits can be calculated.
	\item As a garbage collector, I can report the identifier of a thrown away bottle, so that the responsible consumer can be punished.
\end{enumerate}

\paragraph{Retailer}
\begin{enumerate}[resume, label={\textbf{FR-\protect\twodigits{\theenumi}}},leftmargin=*]
	\item As a retailer, I can report that a consumer has purchased reusable bottles, so that he becomes eligible to receive rewards.
	\item As a retailer, I can report how many reusable bottles a consumer has bought, so that his share of rewards can be calculated. 
	\item As a retailer, I can report which one-way bottles a consumer has bought, so that the ownership of each bottle is recorded.
	\item As a retailer, I can look up the penalty that a specific consumer must pay, so that this amount can be added onto the deposit of each bottle during purchase.
\end{enumerate}

%\subsubsection{Take-back point}
%\begin{enumerate}[resume, label={\textbf{FR-\protect\twodigits{\theenumi}}},leftmargin=*]
%	\item As a tack-back point, I can put in a claim to be reimbursed for making advance refunds, so that I do not experience a loss by offering this service.
%\end{enumerate}
	
\paragraph{Consumer}
\begin{enumerate}[resume, label={\textbf{FR-\protect\twodigits{\theenumi}}},leftmargin=*]  
	\item As a consumer, I can claim a reward for having purchased reusable bottles, so that I will be motivated to so in the future even though one-way packaging is lighter (and more durable) when compared to most reusable bottles. 
\end{enumerate}

\paragraph{Environmental agency}
\begin{enumerate}[resume, label={\textbf{FR-\protect\twodigits{\theenumi}}},leftmargin=*]  
	\item As an environmental agency, I can claim a donation, so that my cause will be better supported through the funding secured.
\end{enumerate}
	
\subsection{Derived Preconditions}
While an incentivized deposit-refund system may be effective, it does assume a variety of conditions that must be in place for it to function properly. As previously hinted, they can be described as the:

\begin{description}[format={\storedescriptionlabel}]
	\item[Obligation to report thrown away bottles]
	\hfill \\
	In addition to reporting the total number of thrown away bottles, garbage collection must also be obliged to communicate the individual identifiers. This assumes that one-way bottles are marked with a label that is unique across all bottles in circulation. Finally, and to automate the whole process, special object recognition machinery will be required along the sorting belt of any given recycling centre.
	\item[Obligation to report bottle purchase]
	\hfill \\
	Retailers and initial distributors alike must report how many and which particular one-way bottles a consumer has purchased. The same (i.e. quantity) goes for buying reusable bottles although this happens through opt-in since the system should not force people to accept rewards. In any case, it will be necessary to have consumers present a unique means of identification as they complete the purchase.
\end{description}

These prevailing conditions could be enforced by amending the system's legal basis (see~\ref{sec:legalBasis}) and then integrating the relevant clauses into the contracts that \ac{DPG} maintains with each participant (see~\ref{sec:DPGAdministration}). However, the exact realization including that of the measures needed as explained above is out of scope (comp.~\ref{sec:goalsScope}) and thus, assumed to be satisfied for the remainder of this paper. 

\subsection{Non-Functional Requirements}



%This revised deposit-refund system is portrayed in \autoref{fig:depositCycleFuture} on \autopageref{fig:depositCycleFuture}.

%\subsection{Blockchain Value Add}
	
%- Usability (Rechenzeit, Storage, Dauer, Kosten)
%- Design Ansätze > Backlog, Wartezeit (Vergleich mit Internetspeed damals), Transaktionsminimierung
%- Public oder Private Blockchain?
%- Abrechnung pro Monat, da ansonsten kein Prozentsatz ermittelt werden kann
%- funktional > User Stories
%- nicht-funktional > schnelle Antwortzeiten, günstige Transaktionen, Skalierbarkeit 
	

\pagebreak

\section{Architecture}

\subsection{Deployment}
\begin{description}
	\item[Public]
	\hfill \\
	
	\item[Private]
	\hfill \\
	
\end{description}


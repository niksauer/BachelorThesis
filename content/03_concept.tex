%!TEX root = ../main.tex

\chapter{Concept}
\label{chp:concept}

\section{Solution Overview}

%- Usability (Rechenzeit, Storage, Dauer, Kosten)
%- Design Ansätze > Backlog, Wartezeit (Vergleich mit Internetspeed damals), Transaktionsminimierung
%- Public oder Private Blockchain?
%- Abrechnung pro Monat, da ansonsten kein Prozentsatz ermittelt werden kann
%- funktional > User Stories
%- nicht-funktional > schnelle Antwortzeiten, günstige Transaktionen, Skalierbarkeit 


An incentivized deposit-refund system, as is outlined in the following, aims to:

\begin{enumerate}[label=(\Alph*)]
  \item \label{bolsterQuote} bolster the quote of beverages sold in reusable packaging
  \item \label{preventPollution} prevent pollution of the environment caused from throwing away bottles
  \item support the cause of environmental agencies
\end{enumerate}

From a game theoretical perspective, \ref{bolsterQuote} and \ref{preventPollution} are presumably achieved by \footnote{Lending itself to the fact that solely threatening retailers and bottlers with the introduction of a deposit-refund system had not been a successful approach (comp. \ref{itm:levyDepositObligation} and \autoref{fig:reusableQuoteDevelopment}).}:

\begin{itemize}
  \item rewarding consumers who purchase reusable bottles
  \item punishing consumers whose bottles are eventually thrown away
\end{itemize}

To translate these measures into reality, it will be necessary to alter the current deposit-refund system (see \ref{sec:depositRefundCycle}) so that:

\begin{itemize}
  \item non-claimed deposits are redistributed among agencies and reusable bottle consumers
  \item a penalty is added onto future deposits of consumers who repeatedly pollute
\end{itemize}

Practically, this will require garbage collection to report how many and which specific bottles have been thrown away. Otherwise, it will not be possible to calculate the amount of non-claimed deposits and identify which consumer was responsible for the bottle at last. The latter infers that bottles must be uniquely identifiable. On the other hand, such a system can only be realized if retailers forward the information that a consumer has bought reusable bottles and has thereby become eligible to receive rewards. Naturally, it should also be prevented that anybody can report these figures, pretend to be an environmental agency or claim someone else's rewards. This core functionality, together with the existing minimum level of service, can be expressed in terms of the different interactions a certain user may have with the system, known as a use case diagram and given by \autoref{fig:useCaseOverview}.

\begin{figure}[hbt]
  \includegraphics[width=\textwidth]{images/concept/use_case_overview}
  \caption{Use Case Overview}
  \label{fig:useCaseOverview}
\end{figure}

\FloatBarrier

%In order to fully realize 
%This revised deposit-refund system is portrayed in \autoref{fig:depositCycleFuture} on \autopageref{fig:depositCycleFuture}.







	
\subsection{Derived Functional Requirements}
	
\subsection{Derived Assumptions}

\subsection{Non-Functional Requirements}

\subsection{Blockchain Value Add}
	
\pagebreak

\section{Architecture}


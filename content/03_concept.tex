%!TEX root = ../main.tex

\chapter{Concept}
\label{chp:concept}

\section{Incentivized Deposit-Refund System}
\label{sec:incentivizedSystem}

\subsection{Overview}
\label{sec:incentivizedOverview}
An incentivized deposit-refund system, as is outlined in the following, aims to:

\begin{enumerate}[label=(\Alph*)]
  \item \label{bolsterQuote} bolster the portion of beverages sold in reusable packaging
  \item \label{preventPollution} prevent pollution of the environment caused by throwing away bottles
  \item support the cause of environmental agencies
\end{enumerate}

From a game theory perspective, \ref{bolsterQuote} and \ref{preventPollution} could be presumably achieved by \footnote{The government's sole threat to introduce a deposit-refund system did not succeeded in achieving \ref{bolsterQuote} (comp.~\autoref{ftn:presumedEffects} and \autoref{fig:reusableQuoteDevelopment} on pp.~\rangeref{presumedHappenedEffects}).}:

\begin{itemize}
  \item rewarding consumers who purchase reusable bottles
  \item penalizing consumers who throw away their bottles
\end{itemize}

To translate these measures into reality, it will be necessary to alter the current deposit-refund system (see~\ref{sec:depositRefundCycle}) so that:

\begin{itemize}
  \item non-claimed deposits are redistributed among agencies and reusable bottle consumers
  \item a penalty is added onto future deposits of consumers who repeatedly pollute
\end{itemize}

Practically, this will require garbage collection agencies to report how many and which specific one-way bottles have been thrown away. Otherwise, it will not be possible to calculate the amount of non-claimed deposits and identify which consumer was last responsible for the bottle. The latter infers that bottles must be uniquely identifiable. On the other hand, such a system can only be realized if retailers forward the information that a consumer has bought reusable bottles and has thereby become eligible to receive rewards. Furthermore, to establish proper ownership, they must also report which one-way bottles a consumer has purchased. Lastly, it should be prevented that anybody can report these figures, pretend to be an environmental agency or claim someone else's rewards. This core functionality, along with the existing minimum level of service, can be expressed in terms of the different interactions a certain user may have with the system. This is depicted in \autoref{fig:useCaseOverview}.

\begin{figure}[hbt]
  \includegraphics[width=\textwidth]{images/concept/use_case_overview}
  \caption{Use case overview}
  \label{fig:useCaseOverview}
\end{figure}

\FloatBarrier

From this overview, it becomes apparent that a revised approach will essentially incorporate an accounting layer to enable the penalization and rewarding of consumers. Basic protocols of interaction (i.e.~buying and returning a bottle) should only be modified to the minimum extent needed.

\subsection{Proposed Rules}
\label{sec:rules}
Having given a high-level introduction to an incentivized deposit-refund system, this section aims to provide an exemplary set of rules that may constitute the underlying business logic. Again, it must be noted that this only serves as a guideline for architecting such a solution. The precise economics, psychological effectiveness and ethics introduced hereby are disregarded (comp.~\ref{sec:goalsScope}).

\subsubsection{Rewards \& Donations}

\begin{itemize}
 	\item 50\% of non-claimed deposits are reserved for donations (\textit{agency fund})
  	\item 50\% of non-claimed deposits are reserved for rewards (\textit{consumer fund})
  	\item rewards and donations can be claimed once within a specific period
  	\item a period's duration is set at 4 weeks
  	\item a consumer's reward ($r_c$) is based on the relative amount of reusable bottles ($b$) he has purchased within the previous period and can thus be calculated as:

\begin{equation}
r_c = \frac{b}{b_{total}} \times f_c
\label{eq:rewardCalculation}
\end{equation}

where $b_{total}$ denotes the total number of reusable bottles sold within that period and $f_c$ resembles the consumer fund

	\item non-claimed rewards are withheld from the consumer by the application % for self-financing
  	\item the agency fund ($f_a$) is evenly distributed, so that each's donation ($d$) will be $d = f_a/n$, where $n$ refers to the number of approved agencies
  	\item non-claimed donations remain available for all agencies in the next period
\end{itemize}

\subsubsection{Penalties}

\begin{itemize}
 	\item if a consumer repeatedly pollutes, he must pay a penalty on top of each deposit
	\item the threshold ($t$) to receive a penalty is set at 5 thrown-away bottles
  	\item the penalty's value ($v$) is set at \EUR{0.05}
  	\item a consumer's penalty ($p_c$) is proportional to the number of bottles ($b$) he has thrown away during the lifetime of this system, meaning that: 

\begin{equation}
p_c = \frac{b - (b \mod t)}{t} \times v
\label{eq:penaltyCalculcation}
\end{equation}

  	\item the penalty is refunded if the consumer returns the bottle on his own
  	\item the penalty is seized if the bottle is thrown away or returned by someone else
%  	\item seized penalties are also used for self-financing
\end{itemize}

% (non-)functional requirements formatter
\def\twodigits#1{%
%  \ifnum#1<100 0\fi
  \ifnum#1<10 0\fi
  \number#1}

% start of functional requirements
\subsection{Functional Requirements}
\label{sec:FRs}
In accordance to \autoref{sec:incentivizedOverview} and \autoref{sec:rules}, the following functional requirements are defined as the bare minimum of features any implementation must provide in order to be used as the accounting backend of an incentivized deposit-refund system. These requirements are expressed from the perspective of an end-user who hopes to achieve different goals by using the system and are grouped by the various roles encountered (comp.~\autoref{fig:useCaseOverview}).

Of course, consumers and environmental agencies will also expect a method to claim the promised rewards and donations respectively, so that two additional requirements are subsequently defined.

\paragraph{Garbage collector}
\begin{enumerate}[font=\sffamily,label={\textbf{FR-\protect\twodigits{\theenumi}}},leftmargin=1.4cm, ref=FR-\protect\twodigits{\theenumi}]
	\item As a garbage collector, I can report the number of thrown-away one-way bottles, so that the amount of non-claimed deposits can be calculated. \label{itm:reportNumber}
	\item As a garbage collector, I can report the identifier of a thrown-away one-way bottle, so that the responsible consumer can be penalized. \label{itm:reportIdentifier}
\end{enumerate}

\paragraph{Retailer}
\begin{enumerate}[resume, font=\sffamily, label={\textbf{FR-\protect\twodigits{\theenumi}}},leftmargin=1.4cm, ref=FR-\protect\twodigits{\theenumi}]
	\item As a retailer, I can report that a consumer has purchased reusable bottles, so that he can become eligible to receive rewards. \label{itm:reportReusablePurchase}
	\item As a retailer, I can report how many reusable bottles a consumer has bought, so that his share of rewards can be calculated. \label{itm:reportReusableNumber}
	\item As a retailer, I can report which one-way bottles a consumer has bought, so that the ownership of each bottle can be recorded. \label{itm:reportOneWays}
	\item As a retailer, I can look up the penalty that a specific consumer must pay, so that this amount can be added onto the deposit of each bottle that he purchases. \label{itm:lookUpPenalty}
\end{enumerate}

\paragraph{Take-back point}
\begin{enumerate}[resume, font=\sffamily, label={\textbf{FR-\protect\twodigits{\theenumi}}},leftmargin=1.4cm, ref=FR-\protect\twodigits{\theenumi}]
%	\item As a tack-back point, I can put in a claim to be reimbursed for making advance refunds, so that I do not experience a loss by offering this service.
	\item As a tack-back point, I can report which one-way bottles a consumer has returned, so that the penalty, if raised, can be refunded or seized accordingly. \label{itm:reportOneWayReturn}
\end{enumerate}

\paragraph{Consumer}
\begin{enumerate}[resume, font=\sffamily, label={\textbf{FR-\protect\twodigits{\theenumi}}},leftmargin=1.4cm, ref=FR-\protect\twodigits{\theenumi}]  
	\item As a consumer, I can claim a reward for having purchased reusable bottles, so that I will be motivated to do so in the future, even though one-way packaging is lighter when compared to most reusable bottles. \label{itm:claimReward}
\end{enumerate}

\paragraph{Environmental agency}
\begin{enumerate}[resume, font=\sffamily, label={\textbf{FR-\protect\twodigits{\theenumi}}},leftmargin=1.4cm, ref=FR-\protect\twodigits{\theenumi}]  
	\item As an environmental agency, I can claim a donation, so that my cause will be better supported through the secured funding. \label{itm:claimDonation}
\end{enumerate}
	
\subsection{Preconditions}
\label{sec:preconditions}
While an incentivized deposit-refund system may be effective, it does assume a variety of conditions that must be put into place for it to function properly. As previously hinted, they can be described as the:

\begin{description}
	\item[Obligation to report thrown-away one-way bottles]
	\hfill \\
	In addition to reporting the total number of thrown-away bottles\hide{(\ref{itm:reportNumber})}, garbage collection must also be obliged to communicate the individual identifiers\hide{(\ref{itm:reportIdentifier})}. This assumes that one-way bottles are marked with a label that is unique across all bottles in circulation. Finally, and to automate the whole process, special object recognition machinery will be required along the sorting belt of any given refuse disposal site.
	\item[Obligation to report bottle purchase]
	\hfill \\
	Retailers and initial distributors alike must report which particular one-way bottles a consumer has purchased\hide{(\ref{itm:reportOneWays})}. The same (i.e. quantity) goes for buying reusable bottles \hide{(\ref{itm:reportReusableNumber})} although this only happens through opt-in as people should not be forced to accept rewards. In both cases, it will be necessary to have consumers present a unique means of identification then they complete their purchase.
	\item[Obligation to levy penalty]
	\hfill \\
	Assuming that the consumer qualifies for a penalty, retailers must be impelled to impose this amount upon purchase of a bottle. This would, in turn, require an extension to cash registers so that the amount can be queried dynamically\hide{(\ref{itm:lookUpPenalty})}, given that the consumer has been priorly identified.
	\item[Obligation to report one-way bottle return]
	\hfill \\
	Similar to reporting a purchase, tack-back points have to report which consumer had been responsible for returning a particular one-way bottle\hide{(\ref{itm:reportOneWayReturn})}. Since the return process can either be automated or executed manually (comp.~\autoref{fig:clearingProcess}), different mechanisms to identify the consumer will be needed.  
\end{description}

These prevailing conditions could be enforced by amending the system's legal basis (see~\ref{sec:legalBasis}) and then integrating the relevant clauses into the contracts that \ac{DPG} maintains with each participant (see~\ref{sec:DPGAdministration}). However, the exact realization including that of the measures needed as explained above is out of the scope of this paper (comp.~\ref{sec:goalsScope}) and thus, assumed to be satisfied for the remainder of this paper. 

\subsection{Non-Functional Requirements}
\label{sec:NFRs}
Whereas functional quality stresses conformance with the design specifications, structural quality addresses non-functional requirements such as security and maintainability \cite[p.~2]{Martinez-Ortiz:2016:QMW:3011141.3011203}. Together, both can be used to constitute the evaluation framework for measuring a system's quality, better known as \ac{QoS}. \citeauthor{Liu:2004:QCP:1013367.1013379} argue that it is not practical to come up with a standard model of attributes because \ac{QoS} is a broad, context-dependent concept \cite[p.~67]{Liu:2004:QCP:1013367.1013379}. Therefore, the following list of desired structural properties is based on those already encompassed in the current deposit-refund system or those which are absolutely necessary to implement an incentivized version.

\paragraph{Access Security}
\begin{enumerate}[font=\sffamily, label={\textbf{NFR-\protect\twodigits{\theenumi}}},leftmargin=1.7cm, ref=NFR-\protect\twodigits{\theenumi}]  
	\item The system shall distinguish between authorized and non-authorized users. More specifically, the individual features outlined above shall only be accessible to their designated user.\label{itm:accessSecurity}
\end{enumerate}

\paragraph{Availability}
\begin{enumerate}[resume, font=\sffamily, label={\textbf{NFR-\protect\twodigits{\theenumi}}},leftmargin=1.7cm, ref=NFR-\protect\twodigits{\theenumi}]  
	\item The system shall be available for use between the hours of 6:00~am and 24:00~pm, which is justified by the fact that most sales and returns happen within this timeframe. Clients must therefore maintain a backlog of unsent reports and transmit them at the next possible point in time.\label{itm:operatingTimes}
\end{enumerate}

\paragraph{Cost}
\begin{enumerate}[resume, font=\sffamily, label={\textbf{NFR-\protect\twodigits{\theenumi}}},leftmargin=1.7cm, ref=NFR-\protect\twodigits{\theenumi}]  
	\item The system shall be no more costly in operation than it currently is. Total membership and bottle registration fees may be taken as a baseline (see~\ref{sec:DPGRoles}).\label{itm:operatingCost}
\end{enumerate}

\paragraph{Scalability}
\begin{enumerate}[resume, font=\sffamily, label={\textbf{NFR-\protect\twodigits{\theenumi}}},leftmargin=1.7cm, ref=NFR-\protect\twodigits{\theenumi}]  
	\item The system shall be able to handle all user requests and by such, shall be scalable to support unlimited growth in the number of actors, reports and claims.\label{itm:unlimitedGrowth}
\end{enumerate}

\paragraph{Efficiency}
\begin{enumerate}[resume, font=\sffamily, label={\textbf{NFR-\protect\twodigits{\theenumi}}},leftmargin=1.7cm, ref=NFR-\protect\twodigits{\theenumi}]  
	\item The system shall execute any request by the end of the day to ensure the timely application of penalties, as well as to have a proper statement ready by the end of the month.\label{itm:responseTime}
\end{enumerate}

\paragraph{Integrity}
\begin{enumerate}[resume, font=\sffamily, label={\textbf{NFR-\protect\twodigits{\theenumi}}},leftmargin=1.7cm, ref=NFR-\protect\twodigits{\theenumi}]  
	\item All monetary amounts (i.e.~rewards and donations) must be calculated accurately before being rounded down to two decimal places. The latter ensures that no more funds are distributed than are available. Additionally, only integer figures may be reported and must be recorded as is.\label{itm:calculcations}
\end{enumerate}

	
\pagebreak

\section{Architecture}
\subsection{Smart-Contract as a Clearing House}
\label{sec:centralClearingHouse}
The biggest roadblock to implementing an incentivized deposit-refund system (see~\ref{sec:incentivizedSystem}) is given by the fact that \ac{DPG} has decided against using a central clearing mechanism to settle refund claims of tack-back points (comp.~\nameref{sec:DPGAdministration} on \autopageref{sec:DPGAdministration}). Instead, refund claimants must invoice the initial distributor who produced the bottle that was taken back, and must then present data proving the number of accepted bottles (comp.~\nameref{sec:depositRefundCycle} on \autopageref{sec:depositRefundCycle}). This design choice is easily explained because the law mandates that: \enquote{a deposit is to be charged by the distributor and each subsequent retailer} (comp.~\nameref{sec:legalBasis} in \ref{sec:legalBasis}). Obviously, transferring the deposit to an escrow account afterwards would not lead to any noticeable benefits compared to a direct settlement. Still, such a change in course will enable an incentivized deposit-refund system to be implemented. Depicted in \autoref{fig:depositCycleFuture} is the transfer of a deposit to an escrow account.

\begin{figure}[hbt]
	  \includegraphics[width=\textwidth]{images/concept/deposit_cycle_future_one_way_token}
	  \caption[Deposit-refund cycle in the future]{Deposit-refund cycle in the future \footnotemark}
	  \label{fig:depositCycleFuture}
\end{figure}

\footnotetext{An extended version detailing the three differently balanced accounting states (comp.~\ref{sec:depositRefundCycle}) can be found in the \autoref{app:concept} on \autopageref{fig:depositCycleFutureExtended}.}

\FloatBarrier

Leveraging Ethereum's decentralized application platform and built-in cryptocurrency (see~\ref{sec:ethereumOverview}), \ac{DPG} could provide participants of the deposit-refund cycle with a fully autonomous method for settling refund claims by deploying a smart-contract that would act as a central clearing house and simultaneously allow for the incentivized structure put forward. A semantic change to the regulatory underpinnings would not be required. 

In order for all this to work, the very first deposit charged for a one-way bottle must be converted to Ether and sent to the smart-contract (step 1a of \autoref{fig:depositCycleFuture})\label{itm:basicRequirement}. At the same time, a non-fungible token (see~\ref{sec:fungibility}) representing the newly introduced bottle must be created using the bottle's unique identifier and associated (step 1b of \autoref{fig:depositCycleFuture}) with the address of the purchaser's \ac{EOA} (see~\ref{sec:accounts}). This means that the only permitted means of identification will be an Ethereum address. Keeping in mind the actor's various obligations (see~\ref{sec:preconditions}), the remainder of the system will be implemented as following: 

\begin{itemize}
	\item [(2)] The deposit of a one-way bottle is still directly paid to the retailer upon each subsequent sale and as already indicated, not again transmitted to the smart-contract. However, should the consumer qualify for a penalty (see~\ref{sec:rules}), this penalty (converted to Ether) will be sent to the smart-contract along with the purchase report itself. Furthermore, each report causes the token to be re-associated with the new consumer. In case the previous owner had to pay a penalty \footnote{Even though unlikely, this indicates that a retailer has lost or thrown away one-way bottles.}, that amount will be credited to him for later withdrawal as he is not responsible for the bottle anymore.
	\item [(3a)] By returning a one-way bottle, the accompanying report triggers the smart-contract to compare the consumer's identity with that recorded for the bottle (i.e.~that associated with the token). Should these match, the penalty will be credited for later withdrawal. Otherwise, the penalty is to be withheld (comp.~\ref{sec:rules}). Irrespective of the outcome, the token is destroyed meaning that the bottle's identifier could be reused in the future. Additionally, tack-back points must physically refund the consumer (comp.~\ref{itm:levyDepositObligation} on \autopageref{itm:levyDepositObligation})
	\item [(3b)] Should the consumer decide to throw away a one-way bottle, this information is (likely) to be reported by garbage collection sooner or later and will trigger an increase in the responsible consumer's record of thrown-away bottles, which is used to calculate his penalty (comp.~\autoref{eq:penaltyCalculcation}). Assuming that a penalty had been paid for this bottle, it is withheld (comp. \ref{sec:rules}). Lastly, the token is burned and the number of bottles which have been thrown away in the current period is incremented, as is required to correctly calculate \autoref{eq:rewardCalculation}.
	\item [(4)] Since tack-back points still directly refund consumers, settlement for these advance payments is needed. This happens by transmitting the same signed dataset which is used today and produced either by a reverse vending machine or a commissioned counting center (see~\ref{sec:depositRefundCycle}). The smart-contract can then verify the payload's signature, after which a message with equivalent value in Ether will be sent to the refund claimant's (i.e. the caller's) address. 
	\item [(5)] Environmental agencies can simply claim their donation by signing a transaction and addressing it to this smart-contract, though, for it to work, the encapsulated message must contain the appropriate function signature (see~\ref{sec:accounts}). Determining that a caller is indeed an environmental agency happens by looking up his approval status. Likewise, eligible consumers (i.e.~those who have been reported to have purchased reusable bottles) can claim their reward in the same fashion with the difference being that permission does not have to be checked since the reward amount is not statically calculated (comp.~\autoref{eq:rewardCalculation}) and by such would simply result in a zero value.
\end{itemize}

\subsection{Component Design}
\label{sec:componentDesign}
Striving for a component-based design represents one of the most important best practices in software engineering because it facilitates developers in maintaining a complex system by decomposing it into parts that are easier to conceive, understand and program. At the same time, the process of dismantling a system should not happen arbitrarily but rather attempt to take the application's domain structure into account to reduce the likelihood of having to remedy large parts afterwards. Accordingly, \autoref{fig:classDiagramConcept} showcases the individual components that have been identified as being essential to realizing the high-level architecture presented beforehand.

\begin{figure}[hbt]
	  \includegraphics[width=\textwidth]{images/concept/class_diagram_concept_short}
	  \caption{Class design model}
	  \label{fig:classDiagramConcept}
\end{figure}

The classes (i.e.~contracts \cite[p.~75]{solidityDocs}) encountered within this (loosely typed) conceptual model can be categorized by the role they take in abstracting the system:

\begin{description}[format={\storedescriptionlabel}]
	\item[Accounting\label{itm:accounting}]
	\hfill \\
	As mentioned in \autoref{sec:incentivizedOverview}, some kind of an accounting layer will be needed to reward reusable bottle consumers. Here, a smart-contract called \texttt{DPGCore} will take upon this task and provide actors with \ref{itm:reportNumber}, \ref{itm:reportReusablePurchase} and \ref{itm:reportReusableNumber}. Since all reported figures are tied to a specific timeframe (i.e.~the current period) and rewards can only be claimed for the past timeframe (comp.~\ref{sec:rules}), \texttt{DPGCore} maintains a reference to two distinct \texttt{Periods} which each track the total number of bottles that have been thrown away in that specific period, as well as the number of reusable bottles a particular \texttt{Consumer} has purchased. The connection to both properties allows \ref{itm:claimReward} and \ref{itm:claimDonation} to be implemented within the same contract. Finally, it makes available two additional functions that for one, enable the settlement process outlined above and two, offer the ability to lock down a deposit that has been transmitted by an initial distributor (comp.~\ref{itm:basicRequirement}). 
	
	By taking a closer look at the conceptual model presented, one can make out the fact that it will not be possible to directly put in a claim to be reimbursed, report a thrown-away bottle or lock up a deposit because the methods are marked as protected. Consequently, they can only be exposed through inheritance. This behavior is intended considering that the penalty can be regarded as an extension to an incentivized deposit-refund system \footnote{As the name implies, incentivization primarily refers to rewarding reusable bottle purchases.} and by such, requires the APIs to be extended with additional arguments that capture the individual bottle identifiers. Further, this design choice will break up the monolith and simplify deployment should it be decided that a penalty is not needed for production, in which case the subclass \texttt{DPGBasic} can be used out of the box.
	\item[Access Control\label{itm:accessControl}]
	\hfill \\
	To relieve \texttt{DPGCore} of the duty to additionally have to track which addresses resemble approved environmental agencies, garbage collectors or tack-back points, \texttt{DPGActorManager} has been established as a separate contract. It is advised that the same or some other key pair in possession of \ac{DPG} is used to deploy this smart-contract, as only the owner (i.e.~the deployer) will be able to add or remove additional \texttt{Actors}. Communication between \texttt{DPGCore} and \texttt{DPGActorManager} will happen by exchanging messages (see~\ref{sec:accounts}). All in all, this will ensure \ref{itm:accessSecurity} in regard to unauthorized donation claims, thrown-away bottle reports and bottle returns. Despite these measures, another, more concerning authorization scheme is still undefined. 
	
	Prevention of unauthorized (reusable \footnote{Non-existent authorization would allow attackers to bluntly increase their share in rewards.}) bottle purchases could be done using at least one of the two following items:
	
	\begin{itemize}
  		\item signed purchase receipt (similar to refund claim)
  		\item list of approved retailer addresses
	\end{itemize}

	Both options would either require a change to existing cashier systems or mandate retailers on all trade levels to register with \ac{DPG} in order to capture their identity, as currently, this can only be done for initial distributors and tack-back points (comp.~\ref{sec:DPGRoles}). On the other hand, option number one would solve two additional problems. First, anybody will be able to verify the report's contents and secondly, it can be ensured that the same purchase will not be reported twice. It is therefore highly recommended that signed receipts be utilized throughout the system, in which case \texttt{DPGActorManager} would only have to maintain a list of approved environmental agencies, provided that no unauthorized party can generate such a receipt.
	
	\item[Penalization\label{itm:penalty}]
	\hfill \\
	Partially discussed above and in \autoref{sec:centralClearingHouse}, a penalty arises from repeatedly detecting that a consumer has thrown away one-way bottles. Detection occurs after garbage collection reports the identifier of a particular bottle, for which then the corresponding token is retrieved along with its owner. This \texttt{DPGToken} will be based on OpenZeppelin's reference implementation of the \texttt{\acs{ERC}-721} non-fungible token standard and even though reuse leads to additional project dependencies, it is still very much desirable because the underlying codebase is regularly tested and community-audited \footnote{\$4.5 Billion worth of digital assets are powered by OpenZeppelin's smart-contracts \cite{openZeppelin}.} \cite{openZeppelinGitHub}. 
	
	By default, OpenZeppelin's \texttt{ERC721Token} contract does not allow users to mint new or burn existing tokens \cite{openZeppelinGitHub}. Therefore, subclassing must be used to expose these protected methods. At the same time, a barrier will be needed to prevent arbitrary usage. Generally, it can be argued that tokens shall only be managed by the application as a result of having received a report. To guarantee this behavior, all function calls within \texttt{DPGToken} will be restricted to \texttt{DPGPenalty}, hereby creating a fully autonomous debt-tracking entity, meaning that no human will be able to transfer their tokens in an attempt to get rid of the return responsibility.
	
	One consequence of separating the penalty's logic from \texttt{DPGCore} (comp.~\ref{itm:accounting} on \autopageref{itm:accounting}) is the need for an additional data structure that tracks how many one-way bottles a particular consumer has thrown away. This structure is given by \texttt{ConsumerStatistic}. Only now, \texttt{DPGPenalty} will be able to implement \ref{itm:reportIdentifier}, \ref{itm:reportOneWays}, \ref{itm:lookUpPenalty} and \ref{itm:reportOneWayReturn}, during which some of the functionality inherited from \texttt{DPGCore} will be re-encapsulated to add the bottle identifier as a function argument. In line with \ref{itm:operatingCost}, it will be possible to specify multiple bottles at once. Transaction cost (see~\ref{sec:gas}) reductions like these should be utilized whenever possible.
\end{description}

%In summary, the components outlined above will lead to five custom build artifacts of which three at most will be deployed \footnote{When a contract inherits from another, only a single contract is created on the blockchain as the code of all base contracts is copied into it \cite[p.~88]{solidityDocs}. Moreover, data structures like the \texttt{Actor} or \texttt{Period} do not constitute their own contract but will be rather realized as simple C-like structs.}. Yet, only two of those are designed for direct user interaction because \texttt{DPGToken} is managed completely autonomously.

%\begin{itemize}
%  \item \ac{DPG} can add/remove agencies and garbage collectors via \texttt{DPGActorManager}
%  \item actors and consumers can report to/withdraw from either \texttt{DPGBasic} or \texttt{DPGPenalty}
%\end{itemize}

\subsection{Conversion of Deposits}
\label{sec:depositConversion}
To this point, it has been assumed that each deposit will be exchanged for Ether of equal value. Considering Ethereum's volatile nature and that of almost any cryptocurrency, this procedure poses a serious concern when thinking about the reward or settlement process. In the worst case, the fluctuation could diminish all of the locked down funds, requiring \ac{DPG} to either reject reimbursement claims or go into debt themselves. Two solutions can be thought of depending on the type of blockchain used in production:

\begin{description}
	\item[Ethereum Main-Chain]
	\hfill \\
	When using Ethereum's public permissionless blockchain, fluctuations are inevitable. However, projects known as stablecoins (e.g.~Tether, MakerDAO, Digix) attempt to create a cryptocurrency of constant value. Many of these are backed by a collateral, pegged to traditional fiat currencies or assets (e.g.~1 gram of gold) and most importantly, are traded as fungible (see~\ref{sec:fungibility}) Ethereum tokens \cite{stablecoins}. Despite each project having its own drawbacks, such a stablecoin could be utilized to better safeguard the funds by: 
	
	\begin{itemize}
  		\item acquiring adequate amount of stablecoin tokens upon receiving a deposit
  		\item reimbursing refund claimants with stablecoin tokens
	\end{itemize}

	Of course, this would happen completely automatically and could be achieved through a separate proxy contract that interacts with a decentralized exchange before returning control to the actual contract method called. 
	
	\item[Private Deployment]
	\hfill \\
	A more predictable approach involves permissioned blockchains, commonly referred to as private blockchains \cite[p.~37]{nist2018}. These are operated by an authorized set of miners who generally use Ether at constant value \cite[p.~36]{nist2018}. Ergo, the solution can be used as is, though it is assumed that:

	\begin{itemize}
  		\item \ac{DPG} operates a public fixed-price exchange at which internal Ether can be sold
  		\item actors and consumers can freely transact and create accounts
	\end{itemize}
	
	The system would then be used with the following peculiarities:
		
	\begin{itemize}
  		\item initial distributors exchange deposit for adequate amount of internal Ether 
  		\item agencies, consumers and claimants exchange their internal Ether when needed
	\end{itemize}
	
	Now, one may suggest that a similar approach could be taken in a public blockchain environment if a second token representing a constant deposit value is used. The problem, however, is that tokens cannot be passed as a function argument. This is due to the fact that tokens are merely a recoding in application state which means that a particular contract has used its storage to remember how many tokens any given address possesses \cite[pp.~181, 193]{Antonopoulos.2018}. At the same time, this is also completely true for Ether itself with the exception being that miners automatically deduct one's balance should they see that a transaction has specified a non-zero positive amount of Ether (comp.~\nameref{sec:stateTransition} on \autopageref{sec:stateTransition}).
\end{description}

\subsection{Design Rationale}
\subsubsection{Accountless}
%The architecture detailed so far has numerous implications as to how that particular implementation of an incentivized deposit-refund system will be used, the most obvious one being that the only permitted means of identification is an Ethereum address. Nonetheless, x
Modeling the entire system as a series of interactions with a smart-contract is a deliberate design choice because, by definition, it prescribes all users to be authenticated with each request. This is due to the fact that only the private-key owning entity can sign transactions for that particular address (comp.~\ref{sec:accounts}) which allows for effortless authorization simply by maintaining a list of approved addresses. More importantly, consumers will not be required to register with yet another service just to leave their bank account details on a centralized server. A plus, considering that such a server is definitely more vulnerable to attacks than the decentralized ownership model embodied by public-key cryptography~\footnote{A secure, non-constantly connected storage medium offers the greatest degree of protection against private-key theft. Devices known as hardware wallets (e.g.~Ledger Nano S, Trezor) have been specifically designed for this purpose \cite[p.~15]{nist2018}.}.

\subsubsection{Withdrawal Pattern}
There are two particular reasons as to why refunded penalties will be credited for later withdrawal instead of being sent to the owner directly or paid out as part of the bottle return process. First, the withdrawal pattern generally represents an Ethereum best practice because it avoids reentrancy attacks \cite[pp.~121, 169]{solidityDocs}. This is also one of the reasons why rewards and donations have to be withdrawn. Secondly, it would be very impractical to incorporate this step into the bottle return process, as reverse vending machines would otherwise have to first look up the penalty and convert it to fiat. Alternatively, the amount could be refunded directly, although this would require tack-back points to keep an additional signed record and submit that record as part of a refund claim.

\subsubsection{Structural Properties}
\label{sec:blockchainValueAdd}
From here, multiple statements will be made as to how a blockchain-based solution performs in respect to the non-functional requirements defined in \autoref{sec:NFRs}. Like before, the type of blockchain (i.e.~public or private) has to be taken into account.

\begin{description}
	\item[Availability (\ref{itm:operatingTimes})]
	\hfill
	\begin{description}[leftmargin=*, wide=0pt]
  		\item[Public]
  		\hfill \\
  		The application will most certainly experience 100\% availability because the likelihood of all 17.6K miners (comp.~\ref{sec:ethereumPopularity}) going offline is extremely low. As such, it will not be necessary for clients to maintain a backlog of unsent reports. Furthermore, the transaction pool will act as a backlog should Ethereum experience an overly high demand in which it is unable to directly process a request.
  		\item[Private]
  		\hfill \\
  		\ac{DPG} will be fully responsible for ensuring continuous uptime although the system's distributed nature will permit multiple nodes to go offline.
	\end{description}

	\item[Costs (\ref{itm:operatingCost})]
	\hfill
	\begin{description}[leftmargin=*, wide=0pt]
  		\item[Public]
  		\hfill \\
  		Assuming a constant gas price and distribution of request types, the operating costs (in Ether) will always be proportional to the application's traffic (i.e.~number of contract invocations) since only the actual computation expended by a miner will be billed (comp.~\ref{sec:gas}). The one-time fee paid for deployment is negligible.
  		\item[Private]
  		\hfill \\
  		\ac{DPG} will have to pay for the infrastructure. Transactions will be free of charge, provided that the yearly membership fees cover the infrastructure costs (comp.~\ref{sec:DPGRoles}). 
	\end{description}
	
	\item[Efficiency (\ref{itm:responseTime})]
	\hfill \\
	Presuming that a miner will not randomly leave out transactions and always prefer transactions with a higher gas price, any request will be executed by the end of the day, if its gas price has been set high enough.	
	
	\item[Integrity (\ref{itm:calculcations})]
	\hfill \\
	Provided that the contract code is error-free, any report will be recorded as is because all transactions are atomic \cite[p.~75]{solidityDocs}. Further, all input arguments are strongly typed \cite[p.~48]{solidityDocs}, so that, if required, only integers will be accepted.
\end{description}
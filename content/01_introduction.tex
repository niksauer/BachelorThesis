%!TEX root = ../main.tex

\chapter{Introduction}

\section{Motivation}
Compared to glass, plastic or aluminum packaging represents a lightweight and durable alternative. The impact of lightweight materials on shipping costs is non-negligible and has therefore been leveraged in the beverage industry for the past 30 years. Simultaneously, the quote of reusable bottles (Mehrwegflasche) has steadily fallen (from 72\% in 1991 \cite[§ 9 Abs. 2]{verpackV} to 45,1\% in 2014 \cite{umweltBundesamt}), which prompted German lawmakers to introduce a system of returnable one-way bottles (Einwegflasche) in 2003 on which a deposit is payable \cite[§ 9 Abs. 2]{verpackV}. 

Contrary to expectations \cite[§ 8]{verpackV} , this regulation has not stopped the influx of one-way bottles but has rather benefitted bottlers. Whenever consumers pollute by leaving behind one-way bottles, an instant 25 cent profit -- assuming that no one else has returned them -- is generated for the producer. This passive profit was estimated to have reached up to 175M\euro{} in 2015 alone \cite{mehrwegSystem}. Ideally, this pollution of the environment should be punished by splitting non-claimed deposits of one-way bottles between environmental agencies and those consumers who regularly purchase reusable bottles, which save more resources. As a further consequence, those consumers who repeatedly neglect to return their one-ways should be required to pay a higher deposit.

This thesis will propose an improved approach which will hopefully maximize the number of returned one-way bottles and effectively steer users towards reusable ones. A prototype simulating this incentivized bottle-return system will be implemented by utilizing smart-contracts and micro-transactions on the blockchain. Further, additional features and benefits that arise from employing this digital solution are to be discussed and drafted accordingly. Finally, it shall be examined whether this system can be applied to other domains, including coffee-to-go cups or fast food containers.

\section{Goals and Scope of Tasks}
%- research design > experiment, i.e. implementation with benchmarking, to conduct research (research methodology)

\section{Thesis Overview}
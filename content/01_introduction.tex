%!TEX root = ../main.tex

\chapter{Introduction}

\section{Motivation}
\label{sec:motivation}
Compared to glass, plastic or aluminum packaging represents a lightweight and durable alternative. The impact of lightweight materials on shipping costs is non-negligible and has therefore been leveraged in the beverage industry for the past 30 years. Simultaneously, the quote of reusable bottles (Mehrwegflasche) has steadily fallen (from 72\% in 1991 \cite[§ 9 Abs. 2]{verpackV} to 45,1\% in 2014 \cite{umweltBundesamt}), which prompted German lawmakers to introduce a system of returnable one-way bottles (Einwegflasche) in 2003 on which a deposit is payable \cite[§ 9 Abs. 2]{verpackV}. 

Contrary to expectations \cite[§ 8]{verpackV}, this regulation has not stopped the influx of one-way bottles but has rather benefitted bottlers. Whenever consumers pollute by leaving behind one-way bottles, an instant 25 cent profit -- assuming that no one else has returned them -- is generated for the producer. This passive profit was estimated to have reached up to 175M\euro{} in 2015 alone \cite{mehrwegSystem}. 

%TODO: highlight that the following is an original idea?

Ideally, this pollution of the environment should be punished by splitting non-claimed deposits of one-way bottles between environmental agencies and those consumers who regularly purchase reusable bottles, which save more resources. As a further consequence, those consumers who repeatedly neglect to return their one-ways should be required to pay a higher deposit. Such a revised approach will hopefully maximize the number of returned one-way bottles and effectively steer users towards reusable ones. Otherwise, a further decline may be inevitable and is shown to have had a direct negative impact on global warming, in addition to the excess amount of waste produced hereof \cite{einwegUmweltbelastung}.

When considered on a case-by-case basis, this problem inherently deals with deposits of very low extrinsic value. Therefore, implementing the proposed approach by utilizing smart-contracts and micro-transactions on the Blockchain may come to mind upon choosing the underlying architecture.

%TODO: any factors that are not taken into account? > ; factors such as the 
%TODO: problem statement clear enough?
%TODO: exclude implementation of client-server model approach as architecture outline should suffice for experienced reader (i.e. 4 semesters of CS)?

\section{Goals and Scope of Tasks}
Since no research on the feasibility of this approach has been undertaken for an application as specific as this one, the study focuses on both a Blockchain implementation as well as that of a traditional client-server model. The objective of the research is to provide a technological recommendation to supporters of an incentivized deposit-refund system and guide \ac{HPE} and interested parties alike in developing decentralized applications on a general basis. The results are relevant as they reduce the go-to-market time of the previously outlined solution to stop the influx of one-way bottles, which in turn reduces pollution and warming of the earth (comp. \ref{sec:motivation}), a serious concern to today's society. Moreover, \ac{HPE} can offer more in-depth Blockchain-related consulting services through the insights gained.

It shall be explicitly noted that the study does not cover analyzing the effectiveness of employing such a system or wether this even represents the best possible approach. Similarly, the legal framework and ethical questions pertaining to its usage are disregarded.

%TODO: cannot ask examples since they make difficult to maintain objectivity as research

%- examples > What are the positives or values of x? How well does x work? How appropriate or desirable is x? What are the advantages and disadvantages of x? > 
%- evaluative questions > enable to provide an opinion or judgement
%- comparative question > explore differences between two or more items 

\section{Research Design}
The main research question is of comparative nature and can be formulated as following: 

\begin{quote}
	"What are the differences and similarities between a Blockchain-based implementation of an incentivized deposit-refund system and that of a traditional client-server model approach?"
\end{quote}

%In order to limit the discussion and ensure a 


%- research design > experiment, i.e. implementation with benchmarking, to conduct research (research methodology)



%- research problem / problem statement
%    - problem definition > define problem you will address (and why) in research
%        - context + background information > who has a problem, when/where does problem arise > what is the cause? 
%        - specificity / scope > what you are trying to solve, what will not be tackled?
%        - relevance > why is important for society/profession to solve problem > what happens if not, who will feel consequences?
%    - problem statement > describe specific problem or issue within larger problem that needs / is relevant enough to be solved 
%        - should indicate what problem is / where it is occurring
%        - must be single problem, explicitly stated, relevant
%- objective > why you are undertaking research, what you are specifically trying to achieve [6]
%    - goal is not to solve a problem, but should identify what study itself will achieve
%- main research question > what you are trying to determine with your study (derived from problem statement) > results are later used to answer problem statement [5]
%    - => example: [2]
%    - what is the basic question you are trying to answer through your research?
%    - break down main question in sub-questions that together answer main question > what more specific questions do you have to answer to be able to answer main research question?
%    - different research question types (descriptive, comparative etc.)
%- brief description of research design
%	- describe study design (part of research plan) > where, when, how (= research method/-ology > survey, experiment etc.), with whom



\section{Thesis Overview}
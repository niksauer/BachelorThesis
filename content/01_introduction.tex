%!TEX root = ../main.tex

%- research problem / problem statement
%    - problem definition > define problem you will address (and why) in research
%        - context + background information > who has a problem, when/where does problem arise > what is the cause? 
%        - specificity / scope > what you are trying to solve, what will not be tackled?
%        - relevance > why is important for society/profession to solve problem > what happens if not, who will feel consequences?
%    - problem statement > describe specific problem or issue within larger problem that needs / is relevant enough to be solved 
%        - should indicate what problem is / where it is occurring
%        - must be single problem, explicitly stated, relevant
%- objective > why you are undertaking research, what you are specifically trying to achieve [6]
%    - goal is not to solve a problem, but should identify what study itself will achieve
%- main research question > what you are trying to determine with your study (derived from problem statement) > results are later used to answer problem statement [5]
%    - => example: [2]
%    - what is the basic question you are trying to answer through your research?
%    - break down main question in sub-questions that together answer main question > what more specific questions do you have to answer to be able to answer main research question?
%    - different research question types (descriptive, comparative etc.)
%- brief description of research design
%	- describe study design (part of research plan) > where, when, how (= research method/-ology > survey, experiment etc.), with whom

\chapter{Introduction}

\section{Motivation}
\label{sec:motivation}
Compared to glass, plastic or aluminum packaging represents a lightweight and durable alternative. The impact of lightweight materials on shipping costs is non-negligible and has therefore been leveraged in the beverage industry for the past 30 years. Simultaneously, the quote of reusable bottles (Mehrwegflasche) has steadily fallen (from 72\% in 1991 \cite[p.~1]{BMU2015} to 43\% in 2015 \cite[p.~4]{UBA2017}), which prompted German lawmakers to introduce a system of returnable one-way bottles (Einwegflasche) in 2003 on which a deposit is paid \cite[p.~53]{Geyer/Smoltczyk2003}. 

Contrary to expectations \hide{\cite[p.~6]{BMU 2010b}} \cite[p.~10]{Hartlep2011Recycling}, this regulation has not stopped the influx of one-way bottles but has rather benefitted bottlers. Whenever consumers pollute by leaving behind one-way bottles, an instant 25 cent profit --- assuming that no one else has returned them~--- is generated for the producer. This passive profit was estimated to have reached up to \EUR{192M} in 2011 alone \cite[p.~245]{PWC2011Mehrweg}. 

Ideally, this pollution of the environment should be punished by splitting non-claimed deposits of one-way bottles between environmental agencies and those consumers who regularly purchase reusable bottles, which save more resources. As a further consequence, those consumers who repeatedly neglect to return their one-ways should be required to pay a higher deposit. Such a revised approach can hopefully maximize the number of returned one-way bottles and effectively steer users towards reusable ones. Otherwise, a further decline may be inevitable and is shown to have had a direct negative impact on global warming, in addition to the excess amount of waste produced hereof \cite{DUHEinweg}.

When considered on a case-by-case basis, this problem inherently deals with deposits of very low extrinsic value. Moreover, such a system has to manage account balances and track the movement of value, increasing the appeal and likelihood for external attacks. Therefore, implementing the proposed approach by utilizing smart-contracts and micro-transactions on the Blockchain may come to mind upon choosing the underlying infrastructure. But how does such an implementation look like; are there special considerations to be made? 

%But can this approach withstand traditional methods; is this endeavour perhaps more desirable? 
%any factors that are not taken into account? > ; factors such as the 

\section{Goals and Scope}
\label{sec:goalsScope}
Since no research on the feasibility of this approach has been undertaken for an application as specific as this one, the study focuses on the end-to-end development process, including design, implementation and deployment. The objective of the research is to guide \ac{HPE} and interested parties alike in developing a mindset suitable for migrating applications onto decentralized platforms and evaluate if this Blockchain-based implementation can withstand the requirements of an incentivized deposit-refund system. The results are relevant as they reduce the go-to-market time of the previously outlined solution to stop the influx of one-way bottles, which in turn reduces pollution and warming of the earth (comp.~\ref{sec:motivation}), a serious concern to today's society. Moreover, \ac{HPE} can offer more profound Blockchain-related consulting services through the insights gained.

It shall be explicitly noted that the study does not cover analyzing the effectiveness of employing such a system (i.e.~reducing the usage of one-way bottles) or verifying wether this represents the best possible approach. Similarly, the legal framework and ethical questions pertaining to its usage are disregarded.

%provide a technological recommendation to supporters of an incentivized deposit-refund system and 

%cannot be evaluative question since it makes it difficult to maintain objectivity as research
%elaborate research methodology (experimental for implementation [desk research], descriptive for architecture [literature review], quantitative for benchmarks [data analysis])

%- evaluative questions > enable to provide an opinion or judgement
%	- examples > What are the positives or values of x? How well does x work? How appropriate or desirable is x? What are the advantages and disadvantages of x? > 
%- comparative question > explore differences between two or more items 

%\section{Research Design}
%The main research question is of descriptive nature and can be formulated as following: 
%
%\begin{quote}
%	"What are the differences and similarities between a Blockchain-based implementation of an incentivized deposit-refund system and that of a traditional client-server model approach?"
%\end{quote}
%
%To limit the variables and length of discussion, an evaluation framework is defined based upon the most important metrics inherent to web services as presented in literature. Additionally, solution specific characteristics such as the prevention of counterfeit claims are considered. Further, both implementations revolve around the same, limited feature set specified later.
%
%\pagebreak
%
%All performance tests (\textit{benchmarking}) are carried out on the following machine:
%
%\begin{table}[hbt]
%	\centering
%	\begin{tabular}{l|l}
%		CPU & 2.5 GHz Intel Core i7 \\
%		\hline
%		GPU & AMD Radeon R9 M370X 2048 MB \\
%		\hline
%		Memory & 16 GB 1600 MHz DDR3 RAM \\
%		\hline
%		Storage & 500 GB SSD \\
%	\end{tabular}
%	\caption{MacBook Pro (Retina, 15-inch, Mid 2015) Technical Specifications~\cite{macbookproSpecifications}}
%\end{table}

% TODO: capitalize autoref chapter

\section{Thesis Overview}
\Cref{chp:theoreticalFramework} reviews the relevant literature and is intended to answer all descriptive research questions that help define the project variables. The key concepts are then applied in \cref{chp:concept} as part of the architectural overview derived from the (non-)functional requirements of an incentivized deposit-refund system given earlier in the chapter. At the same time, the criteria to evaluate the forthcoming implementation are selected. \Cref{chp:implementation} then describes the procedure to implement the solution, after which the actual evaluation may take place in \cref{chp:evaluation}. \Cref{chp:conclusion} uses the results to draw a conclusion, discuss probable alternatives, as well as any limitations encountered during the study. Finally, \cref{chp:summary} summarizes the thesis's goals, methodology and results and is followed by \cref{chp:outlook} to highlight interesting related research opportunities.







%!TEX root = ../main.tex

\chapter{Implementation}
\label{chp:implementation}

\section{Limitations}
The following chapter will attempt to highlight the most important implementation details and explain which considerations have lead to the solution given in \cite{depositRefundGitHub} \footnote{The accompanying class implementation model can be found in the \autoref{app:implementation} on \autopageref{fig:classImplementationModel}.}. This product should by no means be regarded as final since two important aspects are uncovered:

\begin{itemize}
  \item conversion of deposits (see~\ref{sec:depositConversion})
  \item signed receipts (see~\ref{itm:accessControl} on \autopageref{itm:accessControl})
\end{itemize}

Also, it must be noted that all programming has been done in Solidity, a statically typed, high-level language that was influenced by C++, Python and JavaScript \cite[p.~5]{solidityDocs}. Alternatives have not been considered, because Solidity represents Ethereum's most popular language \cite[p.~21]{Antonopoulos.2018}.

\pagebreak

\section{Accounting}

\subsection{A/B Scheme}
In this context, accounting refers to the task of tracking how many reusable bottles a particular consumer has bought in a given timeframe, while also noting down the total number of one-way bottles that have been thrown away in that same timeframe. Component design has delegated this task to \texttt{DPGCore} (comp.~\ref{itm:accounting} on \autopageref{itm:accounting}). 

\Cref{sec:rules} states that the timeframe should be set to 4 weeks which has been done so literally by utilizing one of the six available time unit suffixes \cite[p.~68]{solidityDocs}:

\begin{lstlisting}[caption=Definition of period length, label=lst:periodDataStructure]
uint public constant PERIOD_LENGTH = 4 weeks;
\end{lstlisting}

As previously explained, it will only be necessary to keep an account of the two most recent periods (comp.~\ref{itm:accounting} on \autopageref{itm:accounting}). By such, the application can be thought of as alternating between two data structures, of which one represents the current period used for incoming reports, while the other is stored to determine a consumer's reward. Of course, this will require a reset of the older (i.e.~the reward) period upon switching after 4 weeks. \autoref{fig:accountingTimeline} aims to illustrate this behavior. 

\begin{figure}[hbt]
	  \includegraphics[width=\textwidth]{images/implementation/timeline}
	  \caption{Accounting timeline}
	  \label{fig:accountingTimeline}
\end{figure}

\FloatBarrier

Internally, switching is realized by using a third variable called \texttt{currentPeriodName} that tracks which of the two data structures should be used for new reports. For this task, an enumeration called \texttt{PeriodName} is defined to conveniently limit its range to the two possible cases. Accordingly, the opposite of the current value will then represent the past period. This technique is detailed in \autoref{lst:periodSetup}. 

\begin{lstlisting}[language=Solidity, caption=Declaration of periods and definition of current period tracker, label=lst:periodSetup]
enum PeriodName { A, B}

PeriodName private currentPeriodName = PeriodName.A;
Period private periodA;
Period private periodB;
\end{lstlisting}

In order to avoid repeated logic, two additional helper functions are used to actually retrieve one of the two periods.  \autoref{lst:periodHelperFunction} showcases how this is done in respect to the current period. Again, the inverse logic will yield the past period. 

\begin{lstlisting}[language=Solidity, caption=Helper function to retrieve current period, label=lst:periodHelperFunction]
function getAccountingPeriod() internal view returns (Period storage) {
	return currentPeriod == PeriodName.A ? periodA : periodB;
}
\end{lstlisting}
 
 At this point, it should be noted that a function declared as \texttt{view} promises to not modify the contract's state \cite[p.~82]{solidityDocs} \footnote{\todo[inline]{explain advantage of \texttt{view} functions}}. This commitment can be clearly seen above. Furthermore, the function's return type is marked as \texttt{storage}, meaning that it will return a reference to the period (stored in the contract's storage [see~\ref{sec:accounts}]), not a copy of it. Hereby, persistence is greatly simplified as changes do not have to be written back manually.
 
 Alternatively, one could also directly maintain two storage pointers that reference the data structures in a logical manner. These would then only have to be swapped after 4 weeks. The advantage of this approach is that no helper function is required for retrieval, though it might obscure how the application works and therefore, contradicts with a \ac{dApp}'s goal of establishing trust (comp.~\ref{sec:derivedDefinition}). Still, both possibilities have been confirmed to work equally.

\subsubsection{Period Reset}
Traditionally, resetting a period would be achieved by assigning it a new instance. Sadly, this method is not feasable because the \texttt{Period} data structure uses a mapping to assign each (consumer) address a \texttt{Consumer} data structure as can be seen in \autoref{lst:periodDataStructure}. 

\begin{lstlisting}[language=Solidity, caption=Period data structure, label=lst:periodDataStructure]
struct Period {
	uint index;
	uint reusableBottlePurchases;
	uint thrownAwayOneWayBottles;
	mapping(address => Consumer) consumers;
}
\end{lstlisting}

Like often, this limitation can be traced back to gas (see~\ref{sec:gas}). Considering that mappings are virtually initialized such that every possible key exists and is mapped to a value whose byte-representation is all zeros \cite[p.~62]{solidityDocs}, this behavior is explained by the fact that one cannot provide enough gas to pay for the $2^{256}$ reset operations a miner would have to perform without exceeding the block's \texttt{GASLIMIT} (see~\autoref{ftn:gasLimit}). Therefore, \texttt{delete} has no effect on whole mappings \cite[p.~63]{solidityDocs}.

\begin{lstlisting}[language=Solidity, caption=Consumer data structure, label=lst:consumerDataStructure]
struct Consumer {
	uint reusableBottlesPurchases;
	bool hasClaimedReward;
	uint lastResetPeriodIndex;
}
\end{lstlisting}

A simple workaround is given by \autoref{lst:consumerDataStructure}. Here, an additional property called \texttt{lastResetPeriodIndex} is associated with each consumer to remember the period (index) in which the consumer's attributes were last reset. Should this index be smaller than the contract's \texttt{currentPeriodIndex}, a reset must occur before any attribute may be updated. Logically, the \texttt{lastResetPeriodIndex} must then be set to the \texttt{currentPeriodIndex} to prevent the consumer from being reset each time.

\subsubsection{Period Advancement}

%\noindent\begin{minipage}[b]{.38\textwidth}
%\begin{lstlisting}[language=Solidity, caption=Consumer data structure, label=lst:consumerDataStructure]
%struct Consumer {
%uint reusableBottles;
%bool hasClaimedReward;
%uint lastResetIndex;
%}
%\end{lstlisting}
%\end{minipage}\hfill
%\begin{minipage}[b]{.54\textwidth}
%\begin{lstlisting}[language=Solidity, caption=Period data structure, label=lst:periodDataStructure]
%struct Period {
%uint index;
%uint reusableBottlePurchases;
%uint thrownAwayOneWayBottles;
%mapping(address => Consumer) consumers;
%}
%\end{lstlisting}
%\end{minipage}


%- access control > via DPG 
\subsection{Floating Point Calculations}
\subsection{Deposits \& Voluntary Donations}

\section{Access Control}
\subsection{Ownable}
\subsection{External Interface}

\section{Penalty}
\subsection{ERC-721 Token Standard}
\subsection{OpenZeppelin Framework}
\subsection{Token Transfer Restriction}
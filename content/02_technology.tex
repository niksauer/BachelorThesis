%!TEX root = ../main.tex

\chapter{Theoretical Framework}
\label{chp:theoreticalFramework}

\section{Deposit-Refund System for Bottled Beverages in Germany}
\subsection{Legal Basis}
\label{sec:legalBasis}
Anticipating the depletion of capacity in disposal sites and due to the considerable share in waste generated by packaging \footnote{Recycling rate of packaging was below 50\% in the early 1990s (20\% for aluminium and plastic) \cite{GVM2010}.} \cite{Hartlep2011Recycling}, German federal government enacted an ordinance regarding the avoidance of packaging waste (\textit{Verordnung über die Vermeidung von Verpackungsabfällen}, short: \textit{Verpackungsverordnung} or \textit{VerpackV}) in 1991 which stipulates that packaging \cite[§~1]{verpackV1991}:

\begin{itemize}
  \item must be minimised to an extent necessary for protection and marketing of goods
  \item must be reused where possible
  \item must be recycled if reuse is not applicable 
\end{itemize}

These objectives were supported by introducing the:

\begin{description}
	\item[Obligation to take back packaging]
	\hfill \\
	Producers and distributors of packaging are obliged to take back packaging free of charge, restricted to those goods of type, shape, size and material found within their stock \cite[§§~4-6]{verpackV1991}. Distributors with a retail area of less than 200$m^2$ are further exempted to the same brands. This duty may be only be ignored if a distributor participates in a system which ensures the periodical collection of waste \cite[§~6]{verpackV1991}, known and implemented as a refuse recycling system (\textit{duales System}) in 1990 \cite{Hartlep2011Recycling}.
	\item[Obligation to levy deposits on \gls{beverage packaging}]
	\hfill \\
	A deposit is to be charged by the distributor that will be refunded to the purchaser upon return of the bottle. This duty is applicable on all levels of trade involving domestic beverages sold in non-\gls{reusable packaging} (\textit{one-way}) \cite[§~7]{verpackV1991} and becomes effective as soon as the quote of reusable beverage packaging falls below 72\% \footnote{Aggregated quote derived from weighted average of percentages of reusable packaging encountered across individual segments in 1990 \cite[§~9]{verpackV1991} \cite[p.~134]{Rummler/Schutt 1991}} \cite[§~9]{verpackV1991}.
\end{description}

\todo[inline]{figure showcasing development of aggregated reusable packaging quote}

Starting in 1997, the threshold to levy deposits has been surpassed steadily \cite[p.~1]{BMU 2010a}, requiring an additional assessment of the situation to ensure that the fluctuation does not represent a short-term development \cite[§ 9]{verpackV1991} \cite[p.~5]{Hartlep2011Recycling}. Although obvious at glance, the outcome of a compulsory deposit-refund system only became official on July 2\textsuperscript{nd} 2002 \cite[p.~49]{Geyer/Smoltczyk 2003}, after a lawsuit lead by multiple bottlers and distributors had delayed the initial announcement \cite{spon2011handel}. Introduction of this mandatory system was scheduled for Jan 1\textsuperscript{st} 2003 \cite[p.~53]{Geyer/Smoltczyk 2003}.

\subsection{Amendments}
%updated title to reflect goal of avoiding and recycling \cite{verpackV1998}
The German packaging ordinance underwent several revisions, of which the most important changes affecting the current state shall be highlighted in the following:

\begin{description}
	\item[1998]
	\hfill \\
	Trims deposit obligation to those beverages for which the quote of reusable packaging has fallen when compared to 1991, though still necessitates that overall aggregated quote undercuts threshold of 72\% \cite[pp.~142]{Flanderka 1999}. Even though all five segments (beer, mineral water, carbonated soft drinks, fruit juice/non-carbonated soft drinks and wine) have failed this comparison \cite[p.~1]{BMU 2010a}, juices/non-carbonated soft drinks and wine are exempted because their decline and market volume has not been regarded significant enough to justify the costs introduced with such a system \cite[pp.~1]{BMU 2002} \cite[pp.~6,~9]{Hartlep2011Recycling}. 
	
\todo[inline]{figure showcasing development of reusable packing quote in each segment}

	\item[2005]
	\hfill \\
	Reduces different deposit classifications to single deposit worth \EUR{0.25} valid for all applicable beverages with a filling volume between 0.1 - 3.0 litres. Furthermore, \gls{ecologically advantageous packaging} is admitted the same treatment and classification as that of reusable packaging which represents an important shift of thought since the sole utilisation of packaging has been considered inferior to its reuse with regard to all ecological aspects \cite[p.~1]{BMU 2010c}. Accordingly, the new goal is to promote the use of ecologically advantageous packaging with a target of 80\% market dominance \cite{verpackV2008}. This amendment also adds non-carbonated soft drinks and mixed alcoholic drinks to the list of beverages subject to deposits (irrespective of the reusable packaging quote experienced) while simultaneously protecting dietary products from deposits \cite[p.1408]{BGBl. 2005} \cite[p.~171]{Flanderka/Stroetmann 2009}. Finally, retailers are required to take back any bottles made from a beverage packaging material (glas, metal, paper and plastic) also sold by that store \cite[p.~1]{BMU 2010c}. Previously, return has been limited to bottles of same shape, size [..] and type (comp. \ref{sec:legalBasis}). This regulation attempts to stop isolated deposit-refund systems which arose because discounters created their own specially shaped bottles \cite[p.~168]{Flanderka/Stroetmann 2009}. To conform with this change, industry and commerce established the \ac{DPG}, responsible for setting up a nationwide system \cite[p.~1]{BMU 2010c}.
	
	\item[2008]
	\hfill \\
	Orders distributors to clearly mark bottles as being obliged to a deposit. Further, distributors are demanded to participate in a nationwide deposit-refund system to enable the mutual settlement of claimed deposits \cite[§~9]{verpackV2008}. Lastly, the exemption for dietary products is reduced to infant nutrition. This acts as a measure to counteract increasingly false product declarations attempting to forego deposits \cite[pp. 531]{BGBl. 2008} \cite[p.~171]{Flanderka/Stroetmann 2009}.
\end{description}


\section{Decentralized Applications}

\subsection{Architecture and Components}
\subsection{Platforms}
%\subsection{Tokens}
%\subsection{Ethereum}

\section{Web Service Quality}
